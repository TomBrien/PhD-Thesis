\begin{abstract}
This thesis describes the development and testing of two cold-electron bolometers using highly-doped silicon as the absorber. These detectors exhibit both high sensitivity and low time constants. High sensitivity is achieved due to the weak thermal-link between the electrons and the phonons in the silicon absorber at low temperature ($< 1~\mathrm{K}$). Schottky barriers form naturally between the highly-doped silicon absorber and the superconducting contacts. Selective tunnelling of electrons across these Schottky barriers allows the electron temperature in the silicon absorber to be cooled to below the thermal bath temperature. This direct electron-cooling acts as thermoelectric feedback, reducing the time constant of a cold-electron bolometer to below $1~\mathrm{\upmu s}$. In this work, the underlying physics of these devices is discussed and two devices are presented: one with a highly-doped silicon absorber and the other with strained highly-doped silicon used as the absorber. The design of these detectors is discussed and results are found from numerous characterisation experiments, including optical measurements. These measurements show that a prototype device, using a strained and highly-doped silicon absorber, has a noise-equivalent power of $6.6 \times 10^{-17}~\mathrm{W\,Hz^{\nicefrac{-1}{2}}}$. When photon noise (which dominated this measurement) and noise from the amplifier are disregarded, the underlying device-limited noise-equivalent power is $2.0 \times 10^{-17}~\mathrm{W\,Hz^{\nicefrac{-1}{2}}}$. By measuring the photon noise, the time constant of this detector has been determined to be less than $1.5~\mathrm{\upmu s}$. When compared to the device using unstrained silicon, it is clear that the straining of the silicon absorber, which reduces the electron-phonon coupling, produces a notable improvement in detector performance. Furthermore, a novel amplifier-readout technique, whereby the outputs of two matched amplifiers are cross correlated is introduced; this technique reduces the input-referred amplifier noise from $1~\mathrm{nV\,Hz^{\nicefrac{-1}{2}}}$ to $300~\mathrm{pV\,Hz^{\nicefrac{-1}{2}}}$. 
\end{abstract}
